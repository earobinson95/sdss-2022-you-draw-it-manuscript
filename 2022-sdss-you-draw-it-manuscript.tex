\documentclass[10pt]{article}

\usepackage{sdss2020} % Uses Times Roman font (either newtx or times package)
\usepackage{url}
\usepackage{latexsym}
\usepackage{amsmath, amsthm, amsfonts}
\usepackage{algorithm, algorithmic}  
\usepackage{graphicx}

\usepackage[dvipsnames]{xcolor} % colors
\newcommand{\ear}[1]{{\textcolor{blue}{#1}}}
\newcommand{\svp}[1]{{\textcolor{RedOrange}{#1}}}
\newcommand{\rh}[1]{{\textcolor{Green}{#1}}}
\usepackage[capitalise]{cleveref}
\newcommand\pcref[1]{(\cref{#1})}

\title{`You Draw It': Implementation of visually fitted trends with
\texttt{r2d3}}

%\author{by }
\author{
  Emily A. Robinson \\
  \small{University of Nebraska - Lincoln}\\
  \small{Lincoln, NE} \\
  \small{\tt emily.robinson@huskers.unl.edu} \\\And
 Reka Howard \\
  \small{University of Nebraska - Lincoln}\\
  \small{Lincoln, NE} \\
  \small{\tt rekahoward@unl.edu} \\\And
  Susan VanderPlas \\
  \small{University of Nebraska - Lincoln}\\
  \small{Lincoln, NE} \\
  \small{\tt susan.vanderplas@unl.edu} \\}

\date{}

\begin{document}
\maketitle
\begin{abstract}
How do statistical regression results compare to intuitive, visually
fitted results? Fitting lines by eye through a set of points has been
explored since the 20th century. Common methods of fitting trends by eye
involve maneuvering a string, black thread, or ruler until the fit is
suitable, then drawing the line through the set of points. In 2015, the
New York Times introduced an interactive feature, called `You Draw It,'
where readers are asked to input their own assumptions about various
metrics and compare how these assumptions relate to reality. This
research is intended to implement `You Draw It,' adapted from the New
York Times, as a way to measure the patterns we see in data.
\end{abstract}

{\bf Keywords:} graphics, user interaction, regression

\section{Introduction}

We all use statistical graphics, but how do we know that the graphics we
use are communicating properly? Through experimentation, graphical
testing methods allow researchers to conduct studies geared at
understanding human ability to conduct tasks related to the perception
of statistical charts such as differentiation, prediction, estimation,
and extrapolation. All of these types of tests require different levels
of use and manipulation of the information being presented in the chart.
Efforts in the field of statistical graphics have developed graphical
testing tools and methods, such as the lineup protocol
\cite{buja2009statistical}. The advancement of graphing software
provides the tools necessary to develop new methods of testing
statistical graphics.

\subsection{Measuring Patterns and Trends}

One such aspect of interest is the ability to identify and detect trends
in data. Our visual system is naturally built to look for structure and
identify patterns. For instance, points going down from left to right
indicates a negative correlation between the \(x\) and \(y\) variables.

Initial studies in the 20th century explored the use of fitting lines by
eye through a set of points
\cite{finney1951subjective,mosteller1981eye}. Common methods of fitting
trends by eye involved maneuvering a string, black thread, or ruler
until the fit is suitable, then drawing the line through the set of
points. Recently, Ciccione and Dehaene \shortcite{ciccione2021can}
conducted a comprehensive set of studies investigating human ability to
detect trends in graphical representations from a psychophysical
approach.

In 2015, the New York Times introduced an interactive feature, called
`You Draw It' \cite{aisch2015you}, where readers input their own
assumptions about various metrics and compare how these assumptions
relate to reality. The New York Times team utilizes Data Driven
Documents (D3) that allows readers to predict these metrics through the
use of drawing a line on their computer screen with their mouse.

What remains to be determined is how we can compare our intuitive visual
sense of patterns to those determined by statistical methods.

\subsection{Research Objectives}

The goal of this research is to implement `You Draw It,' adapted from
the New York Times feature, as a way to measure the patterns we see in
data. Here, we provide technical details of the software development,
utilizing interactive graphics in R. We then share results from our
study which validates `You Draw It' as a method for graphical testing
and apply an appropriate data analysis method to the participant data.

\section{Development}

\subsection{`You Draw It' Task}

When completing the graphical task, users are shown an interactive
scatter-plot (Figure 1) along with the prompt, ``Use your mouse to fill
in the trend in the yellow box region.'' The yellow box region moves
along as the user draws their trend-line, providing a visual cue which
indicates where the user still needs to complete a trend line. After the
entire domain has been visually estimated or predicted, the yellow
region disappears, indicating the participant has completed the task.
Prior to study participation, example gifs are shown and users are asked
to complete practice plots to ease the learning curve of the task. Visit
\url{emily-robinson.shinyapps.io/can-you-draw-it} for a test applet.

\begin{figure}[ht]
\begin{center}
\centerline{\includegraphics[width=\columnwidth]{images/ydi-stimuli}}
\caption{You Draw It' task plot as shown to user. \\
\textbf{left:} illustrates what user first sees with the prompt \textit{Use your mouse to fill in the trend in the yellow box region.} \\
\textbf{middle:} illustrates what the user sees while completing the task.\\
\textbf{right:} illustrates the users finished trend line.}
\label{you-draw-it-task-plot}
\end{center}
\end{figure}

\subsection{Code Sketch}

Using Shiny \cite{shinypkg} and JavaScript, we modified the New York
Times `You Draw It' feature for the purpose of testing statistical
graphics allowing us to incorporate user interaction, conduct studies
online, and store participant responses.

Data Driven Documents (D3) \cite{bostock2011d3}, a JavaScript-based
graphing framework that facilitates user interaction, was used to create
the `You Draw It' visual. Major news and research organizations such as
the New York Times, FiveThirtyEight, Washington Post, and the pew
Research Center use D3 to create and customize graphics. A challenge of
working with D3 is the environment necessary to display the graphics and
images. The \texttt{r2d3} package \cite{r2d3pkg} provides an efficient
integration of D3 visuals into R HTML formats. We integrate the D3
visual source code into an R Shiny \cite{shinypkg} application in order
to allow for user interaction and data collection.

Figure 2 illustrates the initial setup of the visual stimuli along with
the iterative process between the user interaction and plotting. We
conducted all data simulation and processing in R and output two data
sets - \emph{point data} and \emph{line data} - containing (x, y)
coordinates corresponding to either a simulated point or fitted value
predicted by a statistical model respectively. Then, the r2d3 package
converts the data sets in R to JavaScript Object Notation (JSON) to be
interpreted by the \texttt{D3.js} code. We define functions in
\texttt{D3.js} to draw the initial plot and set up drawable points for
the user drawn line. Drag events in \texttt{D3.js} are utilized to
observe and react to user input. Shiny Messages are used to communicate
the user interaction between the \texttt{D3.js} code and the R
environment. The plot is then rendered and updated on user interaction
into the R shiny application with the \texttt{RenderD3} and
\texttt{d3Output} functions. Once the user is done drawing the line, we
saved the results of the drawn line to a database for analysis.

\begin{figure}[ht]
\begin{center}
\centerline{\includegraphics[width=\columnwidth]{images/code-sketch-2}}
\caption{Sketch of underlying code for 'You Draw It', illustrating the data simulation conducted in R, the initial setup of the visual stimuli with D3 source code, along with the iterative process between the user interaction and plotting in Shiny. Once the user is done drawing the line, we saved the results of the drawn line to a database for analysis.}
\label{you-draw-it-code-sketch}
\end{center}
\end{figure}

Parameters for aesthetic design choices are defined in a list of options
and \texttt{r2d3} passes these to the D3.js code. For instance, we can
specify the buffer space allowed for the \(x\) and \(y\) axes to avoid
users to anchor their lines to the axes limits. For \texttt{D3.js}
source code, visit
\url{https://github.com/earobinson95/presentations/blob/master/can-you-draw-it/www/main-d3v5.js}.

\subsection{Challenges}

During the development process, there were a few challenges we had to
overcome. We documented these challenges with GitHub commits and outline
places where frustration occurred. Data Driven documents uses Scalable
Vector Graphics (SVG), thus requiring careful transformation between the
pixels and plot coordinates to align the simulated data and the user
drawn line appropriately. With the layered framework of SVG's, it was
important to place the layers in the right order so certain features
would appear where desired. For example, we wanted the opacity of the
yellow box region to appear under the points while still showing the
grid lines. As previously mentioned, \texttt{r2d3} automatically
converts the data set to a JSON file to be interpreted by the underlying
source code, however we provided two data sets (point data and line
data) and converted the two data sets to a JSON file before passing them
to the \texttt{r2d3} argument so that we could reference both data sets
within the \texttt{D3.js} code. One constraint of the development code
to date is that it is only built to work with one-to-one functions and
does not allow for multiple estimates or predictions for one
\(x\)-value.

\section{Application}

\subsection{Tool Validation}

We conducted a study in order to validate `You Draw It' as a method for
graphical testing, comparing results to the less technological method
utilized in Mosteller et al.~\shortcite{mosteller1981eye}. Results from
our study were consistent with those found in the previous study; when
shown points following a linear trend, participants tended to fit the
slope of the first principal component over the slope of the ordinary
least-squares regression line (Figure 3). This trend was most prominent
when shown data simulated with larger variances. This study reinforces
the differences between intuitive visual model fitting and statistical
model fitting, providing information about human perception as it
relates to the use of statistical graphics.

\begin{figure}[ht]
\begin{center}
\centerline{\includegraphics[width=\columnwidth]{images/pca-plot}}
\caption{Comparison between an OLS regression equation which minimizes the vertical distance of points from the line and a regression equation with a slope calculated by the first principal component which minimizes the smallest distance of points from the line.}
\label{pca-plot}
\end{center}
\end{figure}

\subsection{Data Analysis}

Feedback data from conducted studies are collected and stored in a
database for analysis (Figure 4). Within the collected feedback data, we
know the simulated data points, the predicted values from the
statistical model, and the predicted values from the user drawn line. In
our initial studies, a unique data set was simulated independently for
each participant. Therefore, we evaluate the accuracy of the user drawn
line by observing the deviation, vertical residuals, between the user
drawn line and the predicted values from the statistical model. We use a
Generalized Additive Mixed Model (GAMM) to model the vertical residuals
in order to statistically compare visually fitted trends to actual
metrics, simulated data models, or statistical regression results. A
benefit of using a GAMM is the estimation of smoothing splines, allowing
for flexibility in the residual trend.

\begin{figure}[ht]
\begin{center}
\centerline{\includegraphics[width=\columnwidth]{images/eyefitting-trial-plot}}
\caption{Example of three trend lines showing the the OLS fitted, PCA fitted, and participant drawn values overlaid on the simulated data points}
\label{eyefitting-trial-plot}
\end{center}
\end{figure}

\begin{figure}[ht]
\begin{center}
\centerline{\includegraphics[width=\columnwidth]{images/eyefitting-lmer-plot}}
\caption{Estimated trends of residuals (vertical deviation of participant drawn points from both the OLS (blue) and PCA (orange) fitted points) as fit by a linear mixed model.}
\label{eyefitting-lmer-plot}
\end{center}
\end{figure}

\begin{figure}[ht]
\begin{center}
\centerline{\includegraphics[width=\columnwidth]{images/eyefitting-gamm-plot}}
\caption{Estimated trends of residuals (vertical deviation of participant drawn points from both the OLS (blue) and PCA (orange) fitted points) as fit by a generalized additive mixed model.}
\label{eyefitting-gamm-plot}
\end{center}
\end{figure}

\begin{figure}[ht]
\begin{center}
\centerline{\includegraphics[width=\columnwidth]{images/exponential-yloess-spaghetti-plot-2-1}}
\caption{Spaghetti plot of results from a study which asked participants to forcast trends of exponentially increasing data. Participants drawn lines on the linear scale are shown in blue and the log scale are shown in orange. Variability in the statistically fitted regression lines occured due to a unique data set being simulated for each individual; the gray band shows the range fitted values from the statistically fitted regression lines.}
\label{exponential-yloess-spaghetti-plot-2-1}
\end{center}
\end{figure}

\section{Future Work}

In this work, we implemented and validated `You Draw It' as a way to
measure the patterns we see in data. We demonstrated the use of
generalized additive models to statistically model participant data.
While technical details of the development process are presented here,
we intend to create an R package designed for easy implementation of
`You Draw It' task plots in order to make this tool accessible to other
researchers. Further investigation is necessary to implement this method
real data in order to facilitate scientific communication.

\bibliographystyle{sdss2020}
\bibliography{references.bib}
% \bibliography{references}

\end{document}

